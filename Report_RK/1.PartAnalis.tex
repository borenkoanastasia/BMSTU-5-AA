\chapter*{Введение}\label{Input}
\addcontentsline{toc}{chapter}{Введение}


Параллельные вычисления — способ организации компьютерных вычислений,
при котором программы разрабатываются как набор взаимодействующих вычисли-
тельных процессов, работающих параллельно (одновременно).
Такие вычисления обычно реализуются на вычислительных системах, состоящих из множества вычислительных узлов, что 
значительно увеличивает скорость выполнения задачи.  \cite{paral}

В случае, если общую задачу можно разделить на независимые маленькие подзадачи, то параллельные потоки будут работать эффективно 
все время. Однако, если последующие подзадачи зависят от предыдущих, то потоки могут тормозить друг друга. Особенно это заметно, если
один из потоков медленнее других.

Рассморим задачу построения кирпичной стены - стены Фокса. Новый кирпич из ряда выше первого можно положить, только если на местах 
под ним уже лежат 2. Предположим, что в системе 4 вычислительных узла - каменщика. Один из них медленее других. Тогда 3 каменщика 
выложат все кирпичи, которые смогут, из своей области, и будут ждать пока четвертый выложит свою часть. Та же проблема возникнет, 
если производительность каменщиков абсолютно одинакова, но кирпичи (подзадачи) разные по сложности.  Таким образом, стена будет 
строится со скоростью наиболее медленного каменщика. 

Этот показатель можно улучшить, если распланировать нагрузку заранее, так чтобы потоки не простаивали. Распланировать заранее можно,
если известны характеристики потоков и задач.

Однако такие данные о задачах и процессах известны не всегда. Тогда можно планировать нагрузку можно в процессе работы, добавив в 
систему поток-диспетчер, который будет просыпаться через определенные заранее промежутки времени, проверять как справляются потоки,
и регулировать границы их ответственности (их нагрузку), в соответствии с результатами их работы. Такое решение задачи называют 
динамической балансировкой. \cite{res}

Целью данной лабороторной работы является исследование динамической балансировки на примере построения кирпичной стены. Задачами 
данной лабораторной являются:

\begin{enumerate}
  \item реализация алгоритма построения кирпичной стены с балансировкой;
  %\item схемы рассматриваемых алгоритмов;
  \item визуализация работы исслеуемого алгоритма;
  \item описание и обоснование полученных результатов в отчете о выполненной лабораторной работе, выполненного как расчетно-пояснительная 
  записка к работе.
\end{enumerate}

\chapter{Аналитическая часть}\label{Analis}
%\addcontentsline{toc}{chapter}{1 Аналитическая часть}

В данном разделе будет рассмотрен алгоритм построения кирпичной стены с динамической балансировкой нагрузки. 

\section{Алгоритм построения кирпичной стены с диспетчером}\label{BubbleSort}

Построение стены - это укладывани кирпичей в шахматном порядке. Кирпич не может быть уложен в "воздух", т.е. под кирпичем должна быть,
либо земля, либо два других кирпича. 
Каждый каменщик строит свой участок стены, и отчитывается о том, сколько кирпичей он положил, с момента последней проверки диспетчера.
Диспетчер регулярно снимает показания строителей и в соответствии с ними регулирует границы участка стены каждого каменщика.
Следовательно, необходимые потоки для реализации этого алгоритма: 1 поток диспетчер, N-1 - каменщики, где N - количество узлов в 
вычислительной системе.

\section{Вывод аналитической части}\label{End_analis_chapter}

В данном разделе рассмотрен принцип алгоритма построения кирпичной стены с диспетчером. Выделены необходимые для реализации алгоритма
потоки: диспетчер и строители.
 
