
\chapter{Технологическая часть}\label{tecnology}
%\addcontentsline{toc}{chapter}{3 Технологическая часть}

\section{Требования к ПО}\label{Requirements}

Требования к программному обеспечению:
\begin{enumerate}
    \item на вход подается ключ;
    \item результат: значение, соответствующее ключу. 
\end{enumerate}

\section{Выбор языка программирования}\label{Language}

Был выбран язык go, поскольку он удовлетворяет требованиям лабораторной работы. Средой разработки выбрана Visual Studio Code. 
\cite{go}, \cite{linux}, \cite{debian}.

\section{Структуры данных}\label{StructsList}

На листинге \ref{list:matrixstruct}, \ref{list:matrixstruct2} представлено описание структуры кирпича и параметров системы.

\begin{lstinputlisting}
    [caption = {Структура кирпича},
    label = {list:matrixstruct},
    linerange={69-73},
    ]{../rk1_go_bgo1/brick.go}
\end{lstinputlisting}

\newpage

\begin{lstinputlisting}
    [caption = {Параметры системы},
    label = {list:matrixstruct2},
    linerange={40-58},
    ]{../rk1_go_bgo1/brick.go}
\end{lstinputlisting}

\section{Реализация алгоритмов}\label{Listings}

На листинге \ref{list:poslav} представлена реализация алгоритма каменщиков.

\begin{lstinputlisting}
    [caption = {Реализация алгоритма каменщика ч. 1},
    label = {list:poslav},
    linerange={89-98},
    ]{../rk1_go_bgo1/builder.go}
\end{lstinputlisting}
\begin{lstinputlisting}
    [caption = {Реализация алгоритма каменщика ч. 2},
    label = {list:poslav},
    linerange={98-129},
    ]{../rk1_go_bgo1/builder.go}
\end{lstinputlisting}
\begin{lstinputlisting}
    [caption = {Реализация алгоритма каменщика ч. 3},
    label = {list:poslav},
    linerange={129-133},
    ]{../rk1_go_bgo1/builder.go}
\end{lstinputlisting}


На листинге \ref{list:choise} представлена реализация алгоритма диспетчера.

\begin{lstinputlisting}
    [caption = {Реализация алгоритма диспетчера},
    label = {list:choise},
    linerange={121-145},
    ]{../rk1_go_bgo1/dispetcher.go}
\end{lstinputlisting}

Поскольку необходимо визуализировать стену Фокса, для наглядности, один из потоков будет работать в 2 раза медленее - 2 поток)

\section{Тестирование}\label{TestResult}


%\textbf{Модульные тесты}

%В таблице представлены тестовые данные \ref{tab:matrixMultiply}.


%\begin{table}[ht]
%    \caption{Тесты}
%    \centering
%\begin{tabular}{ l | l | l }
%    ${N^{\underline{o}}}$ & Ввод & Вывод   \\ \hline \hline
%    1 &  0   &    Нет элемента.    \\  \hline 
%    2 &  23400005678   &    Kathleen Johnson    \\  \hline 
%    3 &  23400005678   &    2    \\ \hline 
%    4 &  23401515678  &    Donna Weaver    \\  \hline 
%    5 &  23420735678  &    Kenneth Hunter    \\  \hline  
%\end{tabular}
%\label{tab:matrixMultiply}
%\end{table}

В таблице представлены тестовые данные \ref{tab:matrixMultiply}.
\begin{table}[ht]
    \caption{Тесты}
    \centering
\begin{tabular}{ l | l | l }
    ${N^{\underline{o}}}$ & Ввод & Вывод   \\ \hline \hline
    1 &  Стена 1*1   &    4     \\  \hline 
    2 &  Стена 10*10   &    4    \\  \hline 
    2 &  Стена 10*10   &    8    \\  \hline 
    3 &  Стена 10*5   &    1    \\ \hline 
    4 &  Стена 100*100  &    4    \\  \hline 
    5 &  Стена 100*100  &    10    \\  \hline  
\end{tabular}
\label{tab:matrixMultiply}
\end{table}

Тесты пройдены.

%~\section{Оценка трудоемкости}\label{Difficalties}~---
~\section{Вывод технологической части}\label{TechResults}~

Были реализованы исследуемые алгоритмы, программа прошла тесты и удовлетворяет требованиям.

