\chapter*{Введение}\label{Input}
\addcontentsline{toc}{chapter}{Введение}


Конвейер - это машина непрерывного действия, служащая для перемещения сыпучих, кусковых, штучных и др. грузов.

В программировании конвейерная обработка - это один из способов совмещения операций, при котором аппаратура компьютера 
в любой момент времени выполняет одновременно более одной базовой операции. 
Первый способ - параллельные вычисления - уже был приведен раннее. В этой работе будет рассмотрен конвеер.

Целью данной работы является разработка и исследование конвейерных вычислений.
Задачами данной лабораторной являются:
\begin{enumerate}
  %\item изучение алгоритмов сортировки пузырьком, вставками, выбором;
  \item реализация последовательного алгоритма нахождения среднего арифметического матрицы;;
  \item реализация конвейерной обработки на примере шифрования строк;
  \item сравнительный анализ шифрования строк конвеерной обработкой и последовательным алгоритмом;
  \item описание и обоснование полученных результатов в отчете о выполненной лабораторной работе, выполненного как расчетно-пояснительная 
        записка к работе.
\end{enumerate}

\chapter{Аналитическая часть}\label{Analis}
%\addcontentsline{toc}{chapter}{1 Аналитическая часть}

В данном разделе будет рассмотрено понятие конвеерной обработки. Так же будут описаны алгоритмы применяемые в работе для шифрования строк. 

\section{Конвейерная обработка}\label{BubbleSort}

Конвейеризация (или конвейерная обработка) в общем случае основана на разделении подлежащей исполнению функции на более мелкие части, 
называемые ступенями, и выделении для каждой из них отдельного блока аппаратуры. Так обработку любой машинной команды можно разделить на 
несколько этапов (несколько ступеней), организовав передачу данных от одного этапа к следующему. При этом конвейерную обработку можно исполь
зовать для совмещения этапов выполнения разных команд. Производительность при этом возрастает благодаря тому, что одновременно на различных ступенях конвейера выполняются несколько команд. Конвейерная обработка такого рода широко применяется во всех современных быстродействующих
процессорах.

\section{Алгоритмы шифрования строк}\label{ChoiseSort}

Идея шифрования строк - преобразование их для скрытия от посторонних. В то же время, те которым предназночается информация способны
ее дешифровать.
В данной работе будут рассмотрены следующие алгоритмы: шифр Цезаря, шифр XOR, шифр Атбаш.

\textit{Шифр Цезаря}

Имеется ключ от 0 до 26 (для латинского алфавита), каждая буква смещается на значение ключа.

\textit{Шифр Xor}

Строка разбивается на отдельные символы и каждый символ представляется в бинарном виде. Далее с каждым
символом применяется операция XOR с ключом. Результатом является зашифрованная строка.

\textit{Шифр Атбаш}

Каждому i-ому символу алфавита ставится в соответствие символ алфавита (n-i), где n - размер алфавита.

\section{Вывод аналитической части}\label{End_analis_chapter}

В данной работе стоит задача реализации конвейерных вычислений. Были рассмотренны особенности построения конвейерных вычислений.

