\chapter*{Введение}\label{Input}
\addcontentsline{toc}{chapter}{Введение}

В данной лабораторной работе рассматриваются алгоритмы, которые известны своей простотой и необходимостью, алгоритмы сортировки.
Сортировать можно что угодно: числа в массиве, слова в словаре, вещи в магазине и т.д. Наиболее частые сферы применения сортировок в 
программировании
\begin{enumerate}
  \item поиск в массиве по ключу
  \item группировка элементов по одинаковым значениям какого-либо признака
  \item сравнение двух и более массивов на наличие одинаковых элементов
\end{enumerate}

Целью данной лабораторной являются изучение метода динамического программирования на материале алгоритмов сортировки.
Задачами данной лабораторной являются:
\begin{enumerate}
  \item изучение алгоритмов сортировки пузырьком, вставками, выбором;
  \item определение трудоемкостей исследуемых алгоритмов;
  \item реализация указанных алгоритмов;
  \item сравнительный анализ исследуемых алгоритмов;
  \item экспериментальное подтверждение различий в трудоемкости алгоритмов с указанием лучшего и худшего случаев;
  \item описание и обоснование полученных результатов в отчете о выполненной лабораторной работе, выполненного как расчетно-пояснительная 
        записка к работе.
\end{enumerate}

\chapter{Аналитическая часть}\label{Analis}
%\addcontentsline{toc}{chapter}{1 Аналитическая часть}

\section{Сортировка пузырьком}\label{BubbleSort}

Идея алгоритма заключается в следующем: шаг сортировки состоит в проходе от начала к концу по массиву. По пути просматриваются пары 
соседних элементов. Если элементы некоторой пары находятся в неправильном порядке, то меняем их местами. После нулевого прохода в начале
массива окажется самый "легкий" элемент - "пузырек". Следующий проход делается со второго элемента, всплывает 2 "пузырек". И так повторяем,
пока элементы не закончатся. После этого получаем отсортированный массив.


\section{Сортировка выбором}\label{ChoiseSort}

Шаг сортировки - это поиск максимального элемента массива.
В первый проход найденный максимум меняется местами с последним элементом. Во второй проход поиск производится до предпоследнего элемента, 
и найденный элемент меняется местами с предпоследним. И далее, каждый раз перебирается на один элемент меньше и результат ставится на место
с индексом на единицу меньше. Эти действия повторяются до тех пор как массив не закончится. 

\section{Сортировка вставками}\label{InsertSort}

Массив перебирается от начала к концу и каждый элемент обрабатывается по очереди. Слева от очередного элемента будет находиться 
отсортированная часть алгоритма, очередной элемент добавляется в нее так, чтобы не нарушить отсортированность этой части. В 
отсортированной части массива ищется точка вставки для очередного элемента. Сам элемент отправляется в буфер, в результате чего в 
массиве появляется свободная ячейка — это позволяет сдвинуть элементы и освободить точку вставки.

\section{Вывод аналитической части}\label{End_analis_chapter}

В данной работе стоит задача реализации следующих алгоритмов: изучение алгоритмов сортировки пузырьком, вставками, выбором.
Необходимо сравнить алгоритмы умножения матриц по эффективности по времени.
 
